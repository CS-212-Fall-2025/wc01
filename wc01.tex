\documentclass[a4paper]{exam}

\usepackage{amsmath,amssymb, amsthm}
\usepackage{geometry}
\usepackage{graphicx}
\usepackage{hyperref}
\usepackage{xcolor}
\usepackage{wasysym}

\usepackage{xcolor}
\usepackage{wasysym}


\usepackage{tikz}
\usepackage{tikz-qtree}

\usepackage{arabtex}

\usepackage{polyglossia}

\setmainlanguage{english}
\setotherlanguage{urdu}
\newfontfamily\urdufont[Scale=1.25]{jameel.ttf}

\newcommand{\M}{\text{\texturdu{منفرد}}} % Use \M typeset this operator.

\newtheorem{definition}{Definition}
\newtheorem{theorem}{Theorem}

\title{Weekly Challenge 01: Discrete Maths Refresher}
\author{CS 212 Nature of Computation\\Habib University}
\date{Fall 2025}

\boxedpoints


% \printanswers % Uncomment this line

\begin{document}
\maketitle

\begin{definition}
    Let $X$ and $Y$ be two subsets of a set $Z$. The \M operator is defined as, $\M(X,Y) = \{n \in Z \mid (n \in X \land n \not\in Y) \lor  (n \not\in X \land n \in Y)\}$.
\end{definition}

\begin{definition}
    Let $G = (V,E)$ be a graph. A decomposition of $G$ is a list of subgraphs of $G$ (obtained from removing edges of $G$) such that each edge appears in exactly one subgraph in the list.
\end{definition}

\begin{figure}[!h]
    % \vspace*{1cm}
    \begin{center}
    \begin{tikzpicture}[scale=.75]
        
        \node[style={fill=black,circle}] (a) at (4,1.7){};
        \node[style={fill=black,circle}] (b) at (2.5,0.73){};
        \node[style={fill=black,circle}] (c) at (2.939339828,-1.06){};
        \node[style={fill=black,circle}] (d) at (5.06066,-1.06){};
        \node[style={fill=black,circle}] (e) at (5.499,0.73){};
        \draw[black,very thick] (a)--(b) (a)--(c) (a)--(d) (a)--(e) (b)--(c) (b)--(d) (b)--(e) (c)--(d) (c)--(e) (d)--(e);

        \end{tikzpicture} 
        \hspace{1cm} \includegraphics[scale = 0.1]{right-arrow.png} \hspace{1cm}
        \begin{tikzpicture}[scale=.75]
        
            \node[style={fill=black,circle}] (a) at (4,1.7){};
            \node[style={fill=black,circle}] (b) at (2.5,0.73){};
            \node[style={fill=black,circle}] (c) at (2.939339828,-1.06){};
            \node[style={fill=black,circle}] (d) at (5.06066,-1.06){};
            \node[style={fill=black,circle}] (e) at (5.499,0.73){};
            \draw[black,very thick] (a)--(b) (a)--(e) (b)--(c)  (c)--(d) (d)--(e);
    
            \end{tikzpicture} 
            \hspace{1cm}
            \begin{tikzpicture}[scale=.75]
        
                \node[style={fill=black,circle}] (a) at (4,1.7){};
                \node[style={fill=black,circle}] (b) at (2.5,0.73){};
                \node[style={fill=black,circle}] (c) at (2.939339828,-1.06){};
                \node[style={fill=black,circle}] (d) at (5.06066,-1.06){};
                \node[style={fill=black,circle}] (e) at (5.499,0.73){};
                \draw[black,very thick]  (a)--(c) (a)--(d) (b)--(d) (b)--(e) (c)--(e) ;
        
                \end{tikzpicture} 
    \end{center}
        \caption{An example of decomposition of $K_5$ into two $C_5$s.}  
        \label{fig:exampl_ahg}     
\end{figure}


\section*{Problems}
\begin{questions}

  
\question
Let $X$, $Y$ and $Z$ be subsets of some set $S$. Prove or disprove the following claim: If $\M(X,Z) = \M(Y,Z)$ then $X = Y$.
\begin{solution}
    % Enter your solution here 
\end{solution}

\question
Prove that for any $n\in \mathbb{Z}^+\setminus\{1,2\}$ the graph $K_n$ can be decomposed into three isomorphic subgraphs if and only if $n+1$ is not divisible by 3. 
\begin{solution}
    % Enter solution here
\end{solution}

 

\end{questions}
\end{document}

%%% Local Variables:
%%% mode: latex
%%% TeX-master: t
%%% End:



